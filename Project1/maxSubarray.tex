\documentclass[a4paper,10pt]{article}
\usepackage{listings}
\usepackage{color}
\usepackage{algorithm2e}
\usepackage{graphicx}
\usepackage{epstopdf}
\usepackage[margin=1.0in]{geometry}
\usepackage{amsmath,amsthm,amssymb}
 
\definecolor{dkgreen}{rgb}{0,0.6,0}
\definecolor{gray}{rgb}{0.5,0.5,0.5}
\definecolor{mauve}{rgb}{0.58,0,0.82}
 
\lstset{ %
  language=C++,                % the language of the code
  basicstyle=\footnotesize,           % the size of the fonts that are used for the code
  numbers=left,                   % where to put the line-numbers
  numberstyle=\tiny\color{gray},  % the style that is used for the line-numbers
  stepnumber=1,                   % the step between two line-numbers. If it's 1, each line 
                                  % will be numbered
  numbersep=5pt,                  % how far the line-numbers are from the code
  backgroundcolor=\color{white},      % choose the background color. You must add \usepackage{color}
  showspaces=false,               % show spaces adding particular underscores
  showstringspaces=false,         % underline spaces within strings
  showtabs=false,                 % show tabs within strings adding particular underscores
  frame=single,                   % adds a frame around the code
  rulecolor=\color{black},        % if not set, the frame-color may be changed on line-breaks within not-black text (e.g. commens (green here))
  tabsize=2,                      % sets default tabsize to 2 spaces
  captionpos=b,                   % sets the caption-position to bottom
  breaklines=true,                % sets automatic line breaking
  breakatwhitespace=false,        % sets if automatic breaks should only happen at whitespace
  title=\lstname,                   % show the filename of files included with \lstinputlisting;
                                  % also try caption instead of title
  keywordstyle=\color{blue},          % keyword style
  commentstyle=\color{dkgreen},       % comment style
  stringstyle=\color{mauve},         % string literal style
  escapeinside={\%*}{*)},            % if you want to add LaTeX within your code
  morekeywords={*,...}               % if you want to add more keywords to the set
}

% Title Page
\title{Assignment 1: Introduction to Systems Programming}
\author{Kevin Rich, Francis Vo, Soo-Hyun Yoo}

\setlength{\parindent}{0cm}
\setlength{\parskip}{1em}


\begin{document}
	\maketitle

	\section{Mathematical Analysis}
		\subsection{Algorithm 1}
			\begin{algorithm}[H]
				\SetAlgoLined
				\LinesNumbered
				\DontPrintSemicolon
				\KwData{Integer array A of size N}
				\KwResult{Greatest Sum of Subarray}
				\For{$i \gets 0$ \KwTo $N$}{
					\For{$j \gets i$ \KwTo $N$}{
						$s \gets 0$\;
						\For{$k \gets i$ \KwTo $j$}{
							$s \gets s + A[k]$\;
						}
						\If{$s > max$}{
							 $max \gets s$\;
						}
					}
				}
			\caption{Pseudocode for Basic Enumeration}
			\end{algorithm}
			This Algorithm has 3 for-loops so it is O($n^3$).

		\subsection{Algorithm 2}
			\begin{algorithm}[H]
				\SetAlgoLined
				\LinesNumbered
				\DontPrintSemicolon
				\KwData{Integer array A of size N}
				\KwResult{Greatest Sum of Subarray}
				\For{$i \gets 0$ \KwTo $N$}{
					$s \gets 0$\;
					\For{$j \gets i$ \KwTo $N$}{
						 $s \gets A[j]$\;
						 \If{$s > max$}{
							 $max \gets s$\;
						 }
					}
				}
			\caption{Pseudocode for Better Enumeration}
			\end{algorithm}
			This Algorithm has 2 for-loops so it is O($n^2$).

		\subsection{Algorithm 3}
			\noindent {\bf Data}: Integer array A of size N \\
			{\bf Result}: Greatest Sum of Subarray

			\begin{minipage}[!h]{6in}
			\begin{verbatim}

			def MaxSubarray:
			    sums = MaxSubarray_recursive(A)
			    return max(sums)
			\end{verbatim}
			\end{minipage}

			\begin{center}
			\noindent {\bf Algorithm 3.1}: Starting function pseudocode for Divide and Conquer
			\end{center}

			\vspace{1em}
			
			\noindent {\bf Data}: Integer array A of size N \\
			{\bf Result}: Integer array of size 4

			\begin{minipage}[!h]{6in}
			\begin{verbatim}

			def MaxSubarray_recursive:
			    if A.size <= 1:
			        sums.all = A[0]
			        sums.left = A[0]
			        sums.right = A[0]
			        sums.overall = A[0]
			        return sums

			    left_sums = MaxSubarray_recursive(A, left_branch)
			    right_sums = MaxSubarray_recursive(A, right_branch)

			    sums.all = left_sums.all + right_sums.all
			    sums.left = max(left_sums.left, left_sums.all + right_sums.left)
			    sums.right = max(right_sums.right, left_sums.right + right_sums.all)
			    m = left_sums.right + right_sums.left
			    sums.overall = max(sums.all, sums.left, sums.right, m)

			    return sums
			\end{verbatim}
			\end{minipage}

			\begin{center}
			\noindent {\bf Algorithm 3.2}: Recursive function pseudocode for Divide and Conquer
			\end{center}

			\vspace{1em}

			\noindent This algorithm is recursive and decreases by half every step. Each lower step has double the number of calls. Thus, this algorithm is O($n \log n$).


	\newpage
	\section{Theoretical Correctness}

% 		The sum of all the elements in array denoted as sums.all\\
% 		The largest sum starting from the left denoted as sums.left\\
% 		The largest sum starting from the right denoted as sums.right\\
% 		The overall max sum denoted as sums.overall\\
% 
% 		{\bf Base case}: Let $n=1$. Then Sum\_All = A[0], Sum\_Left = A[0], Sum\_Right = A[0], Sum\_Overall = A[0]\\
% 
% 		{\bf Inductive hypothesis}:\\
% 			left_sums = MaxSubarray_recursive(A[0:$n \over 2$-1])\\
% 			right_sums = MaxSubarray_recursive(A[$n \over 2$:n])\\
% 			sums.all = left_sums.all + right_sums.all\\
% 			sums.left = max(left_sums.left, left_sums.all + right_sums.left)\\
% 			sums.right = max(right_sums.right, left_sums.right + right_sums.all)
% 			sums.overall = max(sums.all, sums.left, sums.right, left_sums.right + right_sums.left)
% 
% 		{\bf Proof}:\\
% 			Case 1: Contained entirely in the first half\\
% 				This will be returned as Left.Sum_Overall from the recursive call on the Left.
% 
% 			Case 2: Contained entirely in the second half\\
% 				This will be returned from the recursive call on the Right.
% 
% 			Case 3: Made of a suffix of the first half of maximun sum and the prefix of the second half of the maximum
% 				This will be found using Left.Sum\_Right + Right.Sum\_Left


		{\bf Claim 1}: Given an array $a$ containing $n$ positive integers $a_0, a_1, \dots, a_{n-1}$ for $n > 0$, the divide-and-conquer algorithm (algorithm 3) correctly calculates the sum of the maximum subarray, $\displaystyle s = \max_{i\leq{j}} \left(\sum_{k=i}^j a_k\right),$ for positive integers $i,j<n$.

		{\bf Proof}: As a base case, consider when $n=1$. Then {\tt MaxSubarray\_recursive($n$)} = $a_0$, which is true.

		For the inductive hypothesis, assume that for $n > 1$ and $n \leq q$ for some positive integer $q>1$, the algorithm correctly computes the sum of the maximum subarray.

		Consider an array of size $n=q+1$. Then we can consider one of four cases regarding the location of the maximum subarray within the whole array.

		{\it Case 1}: $\displaystyle s = a$. We correctly capture $s$ in {\tt sums.all}.

		{\it Case 2}: $\displaystyle s = \sum_{k=0}^j a_k$, for $j < q$. We correctly capture $s$ in {\tt sums.left}.

		{\it Case 3}: $\displaystyle s = \sum_{k=i}^q a_k$, for $i > 0$. We correctly capture $s$ in {\tt sums.right}.

		{\it Case 4}: $\displaystyle s = \sum_{k=i}^j a_k$, for $0 < i \leq j < q$. We correctly capture $s$ in {\tt m}.

		In all four cases, we correctly select the maximum sum among {\tt sums.all}, {\tt sums.left}, {\tt sums.right}, and {\tt m} as the sum of the maximum subarray of $a$.

		\begin{center}
		$\Box$
		\end{center}

		
		{\bf Claim 2}: The algorithm terminates.

		{\bf Proof}: Since $n>0$ per the problem statement, $n$ must be at least $1$, and the algorithm returns. This proves the base case.

		For the inductive hypothesis, assume that the algorithm returns for an array of length $n \leq q$ for some positive integer $q>1$. Consider $n=q+1$. The array will be split up into two branches of positive lengths, which means the branches will have lengths less than or equal to $q$. Thus, the algorithm will return for each branch, and the algorithm returns right afterwards.

		\begin{center}
		$\Box$
		\end{center}


		{\bf Claim 3}: The divide-and-conquer algorithm computes the sum of the maximum subarray in O($n \log n$) time.

		{\bf Proof}: Let $n$ be the size of the array of integers, $a$. For $n>1$, the recurrence for the recursive step of the algorithm can be found to be

		\begin{align*}
		T(n) &= \Theta(1) + 2T\left(\frac{n}{2}\right) + \Theta(n) + \Theta(1) \\
		     &= 2T\left(\frac{n}{2}\right) + \Theta(n),
		\end{align*}

		where

		\begin{itemize}
		\item The base case takes $\Theta(1)$,
		\item The recursive calls take $2T\left(\frac{n}{2}\right)$,
		\item The {\tt max()} calculations take $\Theta(n)$, and
		\item The final {\tt return} takes $\Theta(1)$.
		\end{itemize}

		In its entirety, \[T(n) = \begin{cases} \Theta(1) &\mbox{if } n = 1 \\ 2T\left(\frac{n}{2}\right) + \Theta(n) &\mbox{if } n > 1 \end{cases}.\]

		Suppose $T(n) \leq cn \log n + n = \text{O}(n \log n)$. Then

% 		As a base case, let $n=1$. Then $T(1) = c \log 1 + 1= \Theta(1)$, as desired.

% 		For the inductive hypothesis, assume $T(m) \leq cm \log m + m$ for $m < n$. Then if $m = \frac{n}{2}$,
		\begin{align*}
		T(n) &\leq 2\left(c \cdot \frac{n}{2} \log \frac{n}{2}\right) + n \\
		     &\leq cn \log \frac{n}{2} + n \\
		     &= cn \log n - cn \log 2 + n \\
		     &\leq cn \log n \\
		     &= \text{O}(n \log n),
		\end{align*}

		as desired.

		\begin{center}
		$\Box$
		\end{center}


	\section{Testing}

		\begin{tabular}{ | c | c | }
		\hline
		Student ID & Answer\\ \hline
		931678074 & 5703 \\
		930569466 & 8184 \\
		932086449 & 4949 \\
		\hline
		\end{tabular}

	\newpage
	\section{Experimental Analysis}

		\subsection{Algorithm 1}
			\begin{figure}[!htb]
				\centering
				\includegraphics[scale=.5]{timingfiles/alg1plot.png}
			\end{figure}
			\begin{figure}[!htb]
				\centering
				\includegraphics[scale=.5]{timingfiles/alg1plotlog.png}
			\end{figure}

		\newpage
		\subsection{Algorithm 2}
			\begin{figure}[!htb]
			\centering
			\includegraphics[scale=.5]{timingfiles/alg2plot.png}
			\end{figure}
			\begin{figure}[!htb]
			\centering
			\includegraphics[scale=.5]{timingfiles/alg2plotlog.png}
			\end{figure}

		\newpage
		\subsection{Algorithm 3}
			\begin{figure}[!htb]
			\centering
			\includegraphics[scale=.5]{timingfiles/alg3plot.png}
			\end{figure}
			\begin{figure}[!htb]
			\centering
			\includegraphics[scale=.5]{timingfiles/alg3plotlog.png}
			\end{figure}

	\newpage
	\section{Extrapolation and Interpretation}

		\subsection{Extrapolation}
			The functions were calculated using gnuplot's fit function.

		\subsection{Interpretation}
			The functions were calculated using gnuplot's fit function and fitting the data to $f(n) = 10^{m \log_{10}n+c}$
			The slopes for each algorithm is a little lower than the actual power because of the creation overhead of the function has a larger affect on arrays with small sizes.  This will cause the left side to be higher and therefore decreases slope.

		\subsection{Algorithm 1}
			\subsubsection{Extrapolation}
				$f(n) = 4.71599 \times 10^{-10} \times n^3$\\
				$f(n) = 3600 \to n = \boxed{19690}$
			\subsubsection{Interpretation}
				Slope $= \boxed{2.99734}$
		
		\subsection{Algorithm 2}
			\subsubsection{Extrapolation}
				$f(n) = 1.87761 \times 10^{-9} \times n^2$\\
				$f(n) = 3600 \to n = \boxed{1384678}$
			\subsubsection{Interpretation}
				Slope $= \boxed{1.99602}$

		\subsection{Algorithm 3}
			\subsubsection{Extrapolation}
				$f(n) = 1.74832 \times 10^{-8} \times n \times log(n)$\\
				$f(n) = 3600 \to n = 8984428998 = \boxed{8.98 \times 10^9}$
			\subsubsection{Interpretation}
				Slope $= \boxed{1.00506}$

	\newpage
	\section{Code}
		\subsection{Files}
			alg1.cpp - Function for algorithm 1\\
			alg2.cpp - Function for algorithm 1\\
			alg3.cpp - Function for algorithm 1\\
			analysis.cpp - Code to run algorithm and measure times for the number of array then outputs .t file\\
			makefile - To compile files\\
			maxSubarray.pdf - This writeup\\
			maxSubarray.tex - \TeX file for PDF\\
			test.cpp - Allows input of file and runs algorithm on input file\\
			analysis/ - Contains compiled executables for running analysis\\
			test/ - Contains compiled executables for running tests on code, and test array files\\
			timingfiles/ - Contains files for creating plots\\
			timingfiles/*.t - Log of runtimes for different array sizes\\
			timingfiles/*.gp - Code for gnuplot. 2 plots of each algorithm: 1 normal plot, and 1 log-log plot

		\subsection{Algorithm 1}
		\lstinputlisting[language=C++]{alg1.cpp}
		\newpage
		\subsection{Algorithm 2}
		\lstinputlisting[language=C++]{alg2.cpp}
		\newpage
		\subsection{Algorithm 3}
		\lstinputlisting[language=C++]{alg3.cpp}
		

\end{document}
